\documentclass[../main.tex]{subfiles}
\dobib
\begin{document}

\section{Color}

    \subsection{$MCU$}
    The number of malt color units is calculated by using the Lovibond color of the grains
    
        \begin{equation}
            MCU = \frac{\sum_{i}GC_{i}m_{i}}{\vol}
        \end{equation}
        
    where $GC_{i}$ is the grain color of the $i$th grain in degrees Lovibond, $m_{i}$ is the mass of the $i$th grain in pounds, and $\vol$ is the batch volume in gallons.
    
    \subsection{$SRM$}
    The color of the batch is estimated using the Morey formula, which is based on the work of Mosher and Daniels \cite{morey}.
    
        \begin{equation}
            SRM = 1.4922 MCU^{0.6859}
            \label{eq:srm}
        \end{equation}
    
    for $SRM < 50$.

\end{document}