\documentclass[../main.tex]{subfiles}
\dobib
\begin{document}

\section{Specific Gravity Correction}
    This section details the methods for specific gravity correction used in the \texttt{correctSG()} function. The specific gravity reading $SG_{reading}$ from the hydrometer is corrected for temperature by multiplying by the water density correction factor
    
    \begin{equation}
        SG_{corr} = a - b T + c T^{2} - d T^{3}
    \end{equation}
    
    such that
    
    \begin{equation}
        SG = SG_{reading} SG_{corr}
    \end{equation}
    
    where $T$ is the fluid temperature in \tdeg F and the coefficients a-d are given in Table \ref{tab:sgcorr}. This correction applies to original ($OG$) and final ($FG$) gravities, which are then used to calculate alcohol and caloric content. This correction factor is based on a curve fit and should  be accurate between 32$\degree$F and 212$\degree$F \cite{hall_1995}.
    
    \begin{table}[H]
        \centering
        \caption{$SG$ temperature correction coefficients}
        \begin{tabular}{cl}
             Coefficient & \multicolumn{1}{c}{Value} \\
             \hline
             a & 0.00130346 \\
             b & $1.34722124 \times 10^{-4}$ \\
             c & $2.04052596 \times 10^{-6}$ \\
             d & $2.32820948 \times 10^{-9}$
        \end{tabular}
        \label{tab:sgcorr}
    \end{table}

    \textbf{NOTE:} all other sections use the temperature corrected gravities as calculated in this section.
    
\end{document}