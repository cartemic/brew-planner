\documentclass[../main.tex]{subfiles}
\makenomenclature
\begin{document}

\nomenclature{$SG$}{Specific gravity - Density relative to water at STP [-]}
\nomenclature{$OG$}{$SG$ of wort post-boil, prior to fermentation [-]}
\nomenclature{$FG$}{$SG$ post-fermentation [-]}
\nomenclature{$E$}{Extract - amount of material extracted from grains and dissolved into wort [$\degree$P]}
\nomenclature{$RE$}{Real Extract - extract change between initial ($OE$) and final ($AE$) extract values [$\degree$P]}
\nomenclature{$AE$}{Apparent Extract - Extract reading after fermentation, when alcohol is present [$\degree$P]}
\nomenclature{$OE$}{Original Extract - Extract reading prior to fermentation [$\degree$P]}
\nomenclature{$AA$}{Apparent Attenuation - amount of sugars converted to alcohol based on original and apparent extract [\%]}
\nomenclature{$RA$}{Real Attenuation - amount of sugars converted to alcohol based on original and real extract [\%]}
\nomenclature{$ABW$}{Alcohol By Weight [\%]}
\nomenclature{$ABV$}{Alcohol By Volume [\%]}
\nomenclature{$C_{pt}$}{Estimated caloric content per pint of beer [kCal/pt]}
\nomenclature{$T$}{Thermometric temperature [$\degree$F]}
\nomenclature{$SRM$}{Standard Referenc Method - used to compare beer color, based on light attenuation}
\nomenclature{$MCU$}{Malt Color Unit - used to compare beer color for beers under 10.5 SRM}
\printnomenclature

\end{document}