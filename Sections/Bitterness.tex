\documentclass[../main.tex]{subfiles}
\dobib
\begin{document}

\section{Bitterness}
    This section details the methods for IBU estimation used in the \texttt{calcIBU()} function. All equations are from \cite{Palmer}.
    
    \subsection{Alpha Acid Units}
        To calculate the alpha acid units, the formula    
        \begin{equation}
            AAU_{i} = m_{i}A_{i}
        \end{equation}
        is used where $m_{i}$ and $A_{i}$ are the mass and alpha acid percentage of hop variety $i$, respectively.
    
    \subsection{Hop Utilization}
        Tinseth's formulae are used to calculate the time and gravity factors. The time factor for the $i$th hop addition, $f_{t,i}$ is calculated as
        \begin{equation}
            f_{t,i} = \frac{1-\exp(-0.04t_{i}}{4.15}
        \end{equation}
        
        where $t_{i}$ is the boil time of hop addition $i$. The boil gravity factor, $f_{G}$ is calculated as
        
        \begin{equation}
            f_{G} = 1.0256 \left( SG_{boil} - 40 (SG_{boil} - 1)^{2} \right)
        \end{equation}
        
        and is taken to be constant for all hop additions. Using these values, the utilization of the $i$th hop addition, $U_{i}$, is
        
        \begin{equation}
            U_{i} = f_{t,i}f_{G}.
        \end{equation}
    
    \subsection{IBU Calculation}
        The IBUs from the $i$th addition are
        
        \begin{equation}
            IBU_{i} = \frac{74.89 AAU_{i} U_{i}}{\vol}
        \end{equation}
        
        which are then summed to obtain an estimate for overall batch IBUs.
        
        \begin{equation}
            IBU_{batch} = \sum_{i} IBU_{i}
        \end{equation}

\end{document}